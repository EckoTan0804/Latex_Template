%% ==============================
\chapter{\iflanguage{ngerman}{Einleitung}{Introduction}}
\label{sec:Introduction}
%% ==============================

As an useful aid in all scientific work following book is recommended: \cite{deininger1992studienarbeiten}


\subsection{Inline lists}
My robot can:
\begin{enumerate*}[label=(\roman*)]
 \item forward and backward movements,
 \item sidewards movements,
 \item rotation along any curve in space,
 \item place of artificial forces along paths.
\end{enumerate*}

\begin{enumerate*}[label=(\arabic*),itemjoin={{; }}]
    \item the independently controllable wheels
    \item the rechargeable battery pack
    \item the Sick LMS100 laser range scanner
    \item the force-torque sensor
    \item the handlebar for controlling the robotic device
\end{enumerate*}

\url{https://ctan.math.illinois.edu/macros/latex/contrib/enumitem/enumitem.pdf}

You can use inline comments \textbackslash comment\{text to comment\}

\dots
\todo{Rewrite this section}
\nomenclature{IAR-IPR}{Institute for Anthropomatics and Robotics (IAR) - Intelligent Process Control and Robotics (IPR)}
\Blindtext

\subsection{Introduction Sub}
\todo[color=green]{Stuff}
% \todo[due=2017-08-18]{Stuff}
\Blindtext
% \todo[done]{Stuff}
\Blindtext


\subsection{SI Units}
Please use \texttt{siunitx} package for this. See:  \url{https://ctan.org/pkg/siunitx}
