%% Version: 0.2 (31.01.2018)

%% Choose language: english or german
%% Choose Thesis type: seminar, bachelor, master, techreport
%% Use 'declaration' parameter if you want to generate declaration page
%% Use 'final' to disable Todo-notes from final version without deleting each one of them
\documentclass[german,master]{KITthesis}

%% ---------------------------------
%% | Information about the thesis  |
%% ---------------------------------
\title{My Title\\
is Long}
\titleotherlanguage{Mein Titel\\
ist lang}

\author{My Name}
\address{My Address}
\city{7613x Karlsruhe}
\email{my.email@kit.edu}

\keywords{Keywords, of, my, Thesis}
\keywordsotherlanguge{Die, Stichw\"orter, f\"ur, meine, Arbeit}

%% Study program or a seminar/subject
\studyprogram{Intelligente Industrieroboter}

%% Name of your institute (Default: IAR-IPR)
% \institute{}
%% Name of your faculty (Default: KIT-Fakultät für Informatik
% \KITfaculty{}
%% Address of your institute (Default: Engler-Bunte-Ring 8)
% \instituteaddress{}
%% Insitute City (Default: 76131 Karlsruhe)
% \institutecity{}

\reviewerone{Prof. Dr.-Ing. Torsten Kröger}
\reviewertwo{Prof. Dr.-Ing. habil. Björn Hein}
%
% %% The advisors are PhDs or Postdocs
\advisorone{M.Sc. C}
% %% The second advisor can be omitted
\advisortwo{M.Sc. D}
%
% %% Please enter the start end end time of your thesis
\editingtime{xx. Month 20XX}{xx. Month 20XX}

%% --------------------------------
%% | Settings for word separation |
%% --------------------------------
% Help for separation:
% In german package the following hints are additionally available:
% "- = Additional separation
% "| = Suppress ligation and possible separation (e.g. Schaf"|fell)
% "~ = Hyphenation without separation (e.g. bergauf und "~ab)
% "= = Hyphenation with separation before and after
% "" = Separation without a hyphenation (e.g. und/""oder)

% Describe separation hints here:
\hyphenation{
% Pro-to-koll-in-stan-zen
% Ma-na-ge-ment  Netz-werk-ele-men-ten
% Netz-werk Netz-werk-re-ser-vie-rung
% Netz-werk-adap-ter Fein-ju-stier-ung
% Da-ten-strom-spe-zi-fi-ka-tion Pa-ket-rumpf
% Kon-troll-in-stanz
}


%% ------------------------
%% |    Including files   |
%% ------------------------
% Only files listed here will be included!
% Userful command for partially translating the document (for bug-fixing e.g.)
% \includeonly{%
% Content/0-Declaration,
% Content/0-Abstract_EN,
% Content/0-Abstract_DE,
% Content/1-Introduction,
% Content/2-State-of-the-art,
% Content/3-Methods,
% Content/4-Concept,
% Content/5-Implementation,
% Content/6-Results,
% Content/7-Discussion,
% Content/8-Conclusion,
% Content/11-Appendix,
% }

\settitle
%%%%%%%%%%%%%%%%%%%%%%%%%%%%%%%%%
%% Here, main documents begins %%
%%%%%%%%%%%%%%%%%%%%%%%%%%%%%%%%%
\begin{document}

%% Set PDF metadata
\setpdf

%% Title Page
\includetitle

% TODO: Remove this from final version
\includelistoftodos

\includedeclaration

\includeacknowledgments

%% ----------------
%% |   Abstract   |
%% ----------------
%% An abstract both in English
%% and German is mandatory.
%%
%% The text is included from the following files:
%% - Content/0-Abstract_EN
%% - Content/0-Abstract_DE
\includeabstract

%% ------------------------
%% |   Table of Contents  |
%% ------------------------
\inculdetableofcontents

\makenomenclature

%% -----------------
%% |   Main part   |
%% -----------------
\setmainpart

%% ==============================
\chapter{Introduction}
\label{sec:Introduction}
%% ==============================

As an useful aid in all scientific work following book is recommended: \cite{deininger1992studienarbeiten}

\dots
\todo{Rewrite this seciton}
\Blindtext

\subsection{Introduction Sub}
\todo[color=green]{Stuff}
% \todo[due=2017-08-18]{Stuff}
\Blindtext
% \todo[done]{Stuff}
\Blindtext
%% ==============================
\chapter{\iflanguage{ngerman}{Stand der Wissenschenschaft und Technik}{State of the art}}
\label{sec:state_of_the_art}
%% ==============================

\dots



% %% ==============================
% Part is used only in PhD thesis
\part{The ideas}
\chapter{\iflanguage{ngerman}{Methoden}{Methods}}
\label{sec:methods}
%% ==============================

\dots



%% ==============================
\chapter{\iflanguage{ngerman}{Konzept}{Concept}}
\label{sec:concept}
%% ==============================

\dots

Exmaple for a table:

\begin{table}[h]
\centering
\resizebox{\columnwidth}{!}{
 \begin{tabular}{| c | c | c | c |}
  \hline
  Object & Speed $[cm/s]$ & Inner LR $[cm]$ & Inner UR $[cm]$ \\ \hline \hline
  \multirow{3}{*}{\emph{Pitcher}} & real & $ n/a $ & $ 5.65 $ \\
   & $4.60$ & $3.71 \pm 0.67$ & $5.09 \pm 2.23$ \\
   & $10.64$ & $3.55 \pm 0.57$ & $6.14 \pm 0.69$ \\ \hline \hline
  \multirow{3}{*}{Cookie O} & real & $ 7.55 $ & $ 7.55 $ \\
   & $4.60$ & $6.98 \pm 0.27$ & $6.98 \pm 0.27$ \\
   & $10.64$ & $6.77 \pm 0.26$ & $6.77 \pm 0.26$ \\ \hline
 \end{tabular}
 }
\caption{Estimated objects inner parameters on different speeds compared to real sizes of the object. Inner LR stands for lower inner radius of the object and Inner UR for upper radius. The real value of inner lower radius for object \emph{Pitcher} is very hard to be determined precisely, therefore the real value is marked with $n/a$.}
\label{tab:estim_object_parameters_hollow}
\end{table}

%% ==============================
% Part is used only in PhD thesis
\part{The Implementation}
\chapter{\iflanguage{ngerman}{Implementierung}{Implementation}}
\label{sec:implementation}
%% ==============================

\dots
\missingfigure{Please add some figures}



%% ==============================
\chapter{Results}
\label{sec:results}
%% ==============================

\dots


\begin{table}[h]
\centering
\resizebox{\columnwidth}{!}{
 \begin{tabular}{| c | c | c | c |}
  \hline
  Object & Speed $[cm/s]$ & Inner LR $[cm]$ & Inner UR $[cm]$ \\ \hline \hline
  \multirow{3}{*}{\emph{Pitcher}} & real & $ n/a $ & $ 5.65 $ \\
   & $4.60$ & $3.71 \pm 0.67$ & $5.09 \pm 2.23$ \\
   & $10.64$ & $3.55 \pm 0.57$ & $6.14 \pm 0.69$ \\ \hline \hline
  \multirow{3}{*}{Cookie O} & real & $ 7.55 $ & $ 7.55 $ \\
   & $4.60$ & $6.98 \pm 0.27$ & $6.98 \pm 0.27$ \\
   & $10.64$ & $6.77 \pm 0.26$ & $6.77 \pm 0.26$ \\ \hline
 \end{tabular}
 }
\caption{Estimated objects inner parameters on different speeds compared to real sizes of the object. Inner LR stands for lower inner radius of the object and Inner UR for upper radius. The real value of inner lower radius for object \emph{Pitcher} is very hard to be determined precisely, therefore the real value is marked with $n/a$.}
\label{tab:estim_object_parameters_hollow}
\end{table}
%% ==============================
\chapter{\iflanguage{ngerman}{Diskussion}{Discussion}}
\label{sec:discussion}
%% ==============================

\dots



%% ==============================
\chapter{\iflanguage{ngerman}{Zusammensaffung und Ausblick}{Conclusion}}
\label{sec:conclusion}
%% ==============================

\dots




%% --------------------
%% |   Bibliography   |
%% --------------------
\Bibliography{Bibliography/thesis}

%% ----------------
%% |   Appendix   |
%% ----------------
% \cleardoublepage
%% appendix.tex
%%

%% ==============================
\Appendix
\label{ch:Appendix}
%% ==============================



\section{First Appendix Section}
		\label{Anhang-Implementierung}



\begin{figure} [ht]
  \centering
   ein Bild
  \caption{A figure}
  \label{fig:BPMNBeispiela}
\end{figure}


\dots





\end{document}
