\title{My Title\\
is Long}
\titleotherlanguage{Mein Titel\\
ist lang}

\author{My Name}
\address{My Address}
\city{7613x Karlsruhe}
\email{my.email@kit.edu}

\keywords{Keywords, of, my, Thesis}
\keywordsotherlanguge{Die, Stichw\"orter, f\"ur, meine, Arbeit}

%% Study program or a seminar/subject
\studyprogram{Intelligente Industrieroboter}

%% Name of your institute (Default: IAR-IPR)
% \institute{}
%% Name of your faculty (Default: KIT-Fakultät für Informatik
% \KITfaculty{}
%% Address of your institute (Default: Engler-Bunte-Ring 8)
% \instituteaddress{}
%% Insitute City (Default: 76131 Karlsruhe)
% \institutecity{}

\reviewerone{Prof. Dr.-Ing. Torsten Kröger}
\reviewertwo{Prof. Dr.-Ing. habil. Björn Hein}
%
% %% The advisors are PhDs or Postdocs
\advisorone{M.Sc. C}
% %% The second advisor can be omitted
\advisortwo{M.Sc. D}
%
% %% Please enter the start end end time of your thesis
\editingtime{xx. Month 20XX}{xx. Month 20XX}

%% --------------------------------
%% | Settings for word separation |
%% --------------------------------
% Help for separation:
% In german package the following hints are additionally available:
% "- = Additional separation
% "| = Suppress ligation and possible separation (e.g. Schaf"|fell)
% "~ = Hyphenation without separation (e.g. bergauf und "~ab)
% "= = Hyphenation with separation before and after
% "" = Separation without a hyphenation (e.g. und/""oder)

% Describe separation hints here:
\hyphenation{
% Pro-to-koll-in-stan-zen
% Ma-na-ge-ment  Netz-werk-ele-men-ten
% Netz-werk Netz-werk-re-ser-vie-rung
% Netz-werk-adap-ter Fein-ju-stier-ung
% Da-ten-strom-spe-zi-fi-ka-tion Pa-ket-rumpf
% Kon-troll-in-stanz
}

%%
%% --------------------
%% |   Bibliography   |
%% --------------------
\newcommand{\mybibliographyfiles}{Bibliography/ipr_articles,Bibliography/kit_template_example_bibliography,Bibliography/my_thesis_bibliography}


%% --------------------
%% |     Acronyms     |
%% --------------------
\newacronym{ipr}{IAR-IPR}{Institute for Anthropomatics and Robotics - Intelligent Process Control and Robotics}

%% --------------------
%% |     Glossary     |
%% --------------------
\newglossaryentry{robot}
{
    name=robot,
    description={The robot developed in this work.}
}
